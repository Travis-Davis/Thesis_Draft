\label{app:Outline}
\begin{outline}[enumerate]
\1 Chapter 1: Introduction
\2 Discuss Previous Model techniques (specifically AIDJEX era)
\2 Discuss limitations of these models
\3 Poor performance of anisotropy prediction
\3 Frequent parameterizations that are tuned and not physical
\3 Scaling issues
\2 Discuss way forward and development of discrete element models
\2 Discuss DEMSI as transitional state
\3 Ridgid approximation, internal stress = 0, etc
\3 At computationally viable scale, internal stress != 0
\4 Necessitates the development of alternate schemes.
\3 Hopkins' contact model
\2 Discuss my model and why it is being pursued
\3 Scale dependence in Hopkin's contact model not well understood
\3 Observation difficult on floe scale
\3 Develop scheme that provides infrastructure to test behavior of single and multi floe systems
\1 Chapter 2: Description of the Model
\2 Discrete Element Model dynamical equations
\3 Discuss each term in general equation:
\4 Body forces
\4 Skin forces
\4 Internal forces
\3 Introduce Peridynamics as frame work to compute internal forces within system
\4 Introduce Peridynamic equations of motion
\4 Introduce equations governing bond behavior of the system
\4 Describe Elastic-Plastic material and equations used in LAMMPS implementation of system
\3 Introduce model domain and simulation concept.
\4 Scale Requirements and evaluation of simulation parameters:
\4 Bulk Modulus
\4 Shear Modulus
\4 Critical strain
\4 Yield Strength
\4 Time step
\3 Describe model steps
\4 Initiation
\4 Initial Isostasy
\4 Inclusions
\4 Intermediate Isostasy
\4 Fix edges
\4 Integrate to final results
\1 Chapter 3: Simulation Results
\2 TBD
\1 Chapter 4: Hopkin's Parameterization
\2 TBD
\1 Chapter 5: Conclusions
\2 TBD
\1 Appendix 1: Hopkin's Lookup Tables
\2 TBD
\1 Appendix 2: LAMMPS code
\2 TBD
\end{outline}